\documentclass[11pt]{homework}

\newcommand{\hwname}{傅申}
\newcommand{\hwid}{PB20000051}
\newcommand{\hwtype}{计算机网络作业}
\newcommand{\hwnum}{4}

\usepackage{tikz}
\usetikzlibrary{calc,arrows}
\begin{document}
\maketitle
% Your content
\questionnumber{5}
\question{
    \begin{alphaparts}
        \questionpart 如下
        \begin{center}
            \begin{tabular}{|l|c|}
                \hline
                \multicolumn{1}{|c|}{前缀匹配} & 链路接口 \\ \hline
                11100000 00                    & 0        \\ \hline
                11100000 01000000              & 1        \\ \hline
                1110000                        & 2        \\ \hline
                11100001 1                     & 3        \\ \hline
                \multicolumn{1}{|c|}{其他}     & 3        \\ \hline
            \end{tabular}
        \end{center}
        \questionpart{} 第一个地址不与前四个表项匹配, 因此选择链路接口 3.

        第二个地址仅与第 3 个表项匹配, 因此选择链路接口 2.

        第三个地址同时匹配第 3 个和第 4 个表项, 由最长前缀匹配原则, 选择链路端口
        3.
    \end{alphaparts}
}

\questionnumber{7}
\question{
    \begin{center}
        \begin{tabular}{|l|c|c|c|}
            \hline
            \multicolumn{1}{|c|}{前缀匹配} & 目的地址范围         & 地址数量 & 接口 \\ \hline
            1                              & 11000000 到 11011111 & 32       & 0    \\ \hline
            10                             & 10000000 到 10111111 & 64       & 1    \\ \hline
            1                              & 11100000 到 11111111 & 32       & 2    \\ \hline
            \multicolumn{1}{|c|}{其他}     & 00000000 到 01111111 & 128      & 3    \\ \hline
        \end{tabular}
    \end{center}
}

\questionnumber{8}
\question{
    有 $2^{7} > 90 > 2^{6} > 60 > 2^{5} > 2^{4} > 12 > 2^{3}$, 因此可以给子网 2
    分配 223.1.17.0/25 (128 个地址), 给子网 1 分配 223.1.17.128/26 (64 个地址),
    给子网 3 分配 223.1.17.192/28 (16 个地址).
}

\questionnumber{10}
\question{
    \begin{center}
        \begin{tabular}{|l|c|}
            \hline
            \multicolumn{1}{|c|}{前缀匹配} & 链路接口 \\ \hline
            224.0/10                       & 0        \\ \hline
            224.64/16                      & 1        \\ \hline
            224/7                          & 2        \\ \hline
            225.128/9                      & 3        \\ \hline
            \multicolumn{1}{|c|}{其他}     & 3        \\ \hline
        \end{tabular}
    \end{center}
}

\questionnumber{12}
\question{
    \begin{alphaparts}
        \questionpart{} 如下, 其中 a.b.c.d/x-e.f.g.h/y 表示从 a.b.c.d/x 中去掉
        e.f.g.h/y 后的地址范围:
        \begin{center}
            \begin{tabular}{|c|l|c|}
                \hline
                子网 & \multicolumn{1}{|c|}{地址}           & 地址数量 \\ \hline
                A    & 214.97.254/24                        & 256      \\ \hline
                B    & 214.97.255.0/25                      & 128      \\ \hline
                C    & 214.97.255.128/25--214.97.255.248/29 & 120      \\ \hline
                D    & 214.97.255.248/31                    & 2        \\ \hline
                E    & 214.97.255.250/31                    & 2        \\ \hline
                F    & 214.97.255.252/30                    & 4        \\ \hline
            \end{tabular}
        \end{center}
        \questionpart{} 路由器 R1 连接 A, D, F, 转发表为
        \begin{center}
            \begin{tabular}{|l|c|}
                \hline
                \multicolumn{1}{|c|}{前缀匹配}     & 链路接口 \\ \hline
                11010110 01100001 11111110         & A        \\ \hline
                11010110 01100001 11111111 1111100 & D        \\ \hline
                11010110 01100001 11111110 111111  & F        \\ \hline
            \end{tabular}
        \end{center}
        路由器 R2 连接 C, E, F, 转发表为
        \begin{center}
            \begin{tabular}{|l|c|}
                \hline
                \multicolumn{1}{|c|}{前缀匹配}     & 链路接口 \\ \hline
                11010110 01100001 11111111 1       & C        \\ \hline
                11010110 01100001 11111111 1111101 & E        \\ \hline
                11010110 01100001 11111110 111111  & F        \\ \hline
            \end{tabular}
        \end{center}
        路由器 R3 连接 B, D, E, 转发表为
        \begin{center}
            \begin{tabular}{|l|c|}
                \hline
                \multicolumn{1}{|c|}{前缀匹配}     & 链路接口 \\ \hline
                11010110 01100001 11111111 0       & B        \\ \hline
                11010110 01100001 11111111 111110  & D        \\ \hline
                11010110 01100001 11111110 1111101 & E        \\ \hline
            \end{tabular}
        \end{center}
    \end{alphaparts}
}

\questionnumber{14}
\question{
    假定数据报的首部长度为 20 字节.

    需要发送 2400 字节的数据报, 数据长度为 2380. MTU 为 700 字节, 则每个报文中数
    据长度最大为 680 字节. 因此会生成 $\lceil \frac{ 2380 }{ 680 }\rceil = 4$ 个
    分片, 四个分片的标识号均为 422, 长度分别为 700, 700, 700, 360 字节, 标志分别
    为 1, 1, 1, 0, 片偏移分别为 0, 85, 170, 255.
}
\end{document}
